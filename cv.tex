%%%%%%%%%%%%%%%%%%%%%%%%%%%%%%%%%%%%%%%%%
% Plasmati Graduate CV
% LaTeX Template
% Version 1.0 (24/3/13)
%
% This template has been downloaded from:
% http://www.LaTeXTemplates.com
%
% Original author:
% Alessandro Plasmati (alessandro.plasmati@gmail.com)
%
% License:
% CC BY-NC-SA 3.0 ( )
%
% Important note:
% This template needs to be compiled with XeLaTeX.
% The main document font is called Fontin and can be downloaded for free
% from here: http://www.exljbris.com/fontin.html
%
%%%%%%%%%%%%%%%%%%%%%%%%%%%%%%%%%%%%%%%%%

%----------------------------------------------------------------------------------------
%	PACKAGES AND OTHER DOCUMENT CONFIGURATIONS
%----------------------------------------------------------------------------------------

\documentclass[a4paper,10pt]{article} % Default font size and paper size

\usepackage{fontspec} % For loading fonts
\defaultfontfeatures{Mapping=tex-text}

\usepackage{xunicode,xltxtra,url,parskip} % Formatting packages

\usepackage[usenames,dvipsnames]{xcolor} % Required for specifying custom colors

\usepackage[big]{layaureo} % Margin formatting of the A4 page, an alternative to layaureo can be \usepackage{fullpage}
% To reduce the height of the top margin uncomment: \addtolength{\voffset}{-1.3cm}

\usepackage{hyperref} % Required for adding links	and customizing them
\definecolor{linkcolour}{rgb}{0,0.2,0.6} % Link color
\hypersetup{colorlinks,breaklinks,urlcolor=linkcolour,linkcolor=linkcolour} % Set link colors throughout the document

\usepackage{titlesec} % Used to customize the \section command
\titleformat{\section}{\Large\scshape\raggedright}{}{0em}{}[\titlerule] % Text formatting of sections
\titlespacing{\section}{0pt}{3pt}{3pt} % Spacing around sections

\titleformat{\subsection}{\scshape\raggedright\large}{}{1em} {} []



\begin{document}

\pagestyle{empty} % Removes page numbering

%----------------------------------------------------------------------------------------
%	NAME AND CONTACT INFORMATION
%----------------------------------------------------------------------------------------

\par{\centering{\Huge André \textsc{Videla}}\bigskip\par} % Your name
\par{\centering{\textsc{iOS developper / Functional Programmer}}\bigskip\par} % Your name
\section{Informations personnelles }

\begin{tabular}{rl}
\textsc{Date et lieu de naissance:} & 20 August 1991, Suisse \\
\textsc{Adresse:} & Rue pré-du-marché 9B, 1004 Lausanne, Suisse \\
\textsc{Telephone:} & +41 76 822 0541\\
\textsc{email:} & \href{mailto:andre.videla@gmail.com}{andre.videla@gmail.com}
\end{tabular}

%----------------------------------------------------------------------------------------
%	Bio
%----------------------------------------------------------------------------------------

\section{Bio}
Passionné par les languages de programmation, je recherche l'excellence, la perfection ainsi que la beauté dans le code. Une large experience dans le monde du GameDev m'a appris à mélanger élégance et pragmatisme, une leçon que j'applique au quotidiens dans mes projets iOS. J'ai également un grand intérêt pour la théorie et conception de languages formels.
%----------------------------------------------------------------------------------------
%	WORK EXPERIENCE 
%----------------------------------------------------------------------------------------

\section{Expérience professionnelle }

\begin{tabular}{r|p{9.3cm}|l}
\textsc{Juin 2017} & Software Engineer at Sicpa, Lausanne\\
\textsc{Oct. 2016} & Mobile engineer for iOS (CDD)\\ 
& \footnotesize{J'ai conçu, développé et maintenu des API publiques ainsi que des libraires internes. J'ai également développé plusieurs petites applications et maintenu de gros projets iOS à l'étranger. Tout cela dans un environnement agile Scrum.} & Swift, Obj-C\\
\\
\textsc{Mars 2016}  & Sortie du jeu \href{https://arcaea.lowiro.com}{Arcaea} pour Android et iOS par lowiro (Remote)& \\
& \footnotesize{J'ai conçu et implémentée un language dédié à la représentation des notes d'une chanson dans le jeu Arcea. L'implémentation à été faite en C++ pour la partie client mobile et en Javascript pour l'éditeur web.} & C++, ES6\\
\\
%------------------------------------------------
\textsc{Sept 2015} & Sortie de l'application \href{https://itunes.apple.com/us/app/hackerspaces/id1035583993?ls=1&mt=8}{Hackerspaces} pour iOS &\\
& \footnotesize{J'ai développé tous les aspects de l'application: prototyping, UI, UX, tests unitaires, tests fonctionnels, déploiement sur l'App Store, intégration continue et support de l'application à travers toutes les versions de Swift.} & Swift\\
\\
\textsc{Mai 2015} & Sortie du jeu T-Aiko pour Android et iOS (Remote) & \\
\textsc{Mai 2014} & \footnotesize{J'ai collaboré avec le développeur principal de T-aiko pour implémenter les outils de parsing des chansons, la gestion des évènement en temps réel et la gestion de l'interface utilisateur.} & Java\\
\\
\textsc{Fev. 2014} & Développeur volontaire chez Dischan (Remote)\\
\textsc{Sept 2012} & \footnotesize{J'étais responsable du port Android du moteur de jeu utilisé par Dischan. J'ai implémenté les outils de parsing et d'interprétation d'un language dédié au scripting (ressemblant a du python), la gestion des assets, la gestion de l'audio, la composition des scenes en 2D et leur rendu ainsi que la gestion de l'interface utilisateur. De plus je faisais partie de l'équipe QA pour les autres projets de l'équipe.} & Java
\end{tabular}

%----------------------------------------------------------------------------------------
%	EDUCATION
%----------------------------------------------------------------------------------------

\section{Education}

\begin{tabular}{rl}	

Jui. 2017 & EUTypes summer school \\&\footnotesize{Coq, Agda, HoTT, $\lambda$-calculus} \\
\multicolumn{2}{c}{}\\
Mai 2017 & Introduction to Logic by Stanford University on Coursera. \footnotesize{\href{https://www.coursera.org/account/accomplishments/certificate/RPGEPLA94HFF}{statement of accomplishment}}\\
\multicolumn{2}{c}{}\\
2012 - 2016 & Suivi un Bachelor en Computer Science a l'EPFL, Lausanne\\
& \footnotesize{Obtenu 131 credits}\\
\multicolumn{2}{c}{}\\

%------------------------------------------------

2007 - 2012 & Maturité Gymnasiale\\ & option principale Physique et Application des Math\\ & option secondaire Introduction à la programmation\\
\end{tabular}

\renewcommand{\arraystretch}{1.2}

%----------------------------------------------------------------------------------------
%	LANGUAGES
%----------------------------------------------------------------------------------------

\section{Langues}

\begin{tabular}{rl}
\textsc{Français:} & Langue maternelle\\

\textsc{Anglais:} & Parlé et écrit courament\\

\textsc{Espagnol:} & Connaissances orales\\

\textsc{Japonais:} & Connaissances basiques\\

\end{tabular}

%----------------------------------------------------------------------------------------
%	Programming Languages
%----------------------------------------------------------------------------------------

\section{Languages de programmation}

\begin{tabular}{r|l}
Maîtrise & Swift, Java\\
Excellence & Scala, JS, Obj-C\\
Intermédiaire & Idris, Haskell, Clojure, C++, C, C\#, \LaTeX\\
Débutant & Coq, Rust, Agda\\
\end{tabular}
%----------------------------------------------------------------------------------------
%	INTERESTS AND ACTIVITIES
%----------------------------------------------------------------------------------------

\section{Activités et hobbies}
\begin{itemize}
	\item \textsc{Jeux Video} Mon expérience passée dans les jeux video compétitifs m'ont appris a collaborer au sein d'une équipe, gérer des situations imprévues et rester lucide sous la pression. Actuellement, j'organise un "club jeux video" (similaire a un club de lecture) mensuel avec un petit groupe où nous discutons les propriétés narratives de nos experiences interactives.
	\item \textsc{Math} J'étudie continuellement des derniers développements théoriques dans le développement de languages de programmation et la logique formelle. Ces connaissance de pointe assouvissent mon désir d'apprendre et m'aident à développer des solutions originales et efficaces dans mes projets de tous les jours.
	\item \textsc{Sport} Je pratiques plusieurs sports individuels comme la course a pied, le fitness et la natation afin de garder ma forme physique. Je participe volontiers aux activités en commun comme par exemple rejoindre le groupe de badminton chez Sicpa.
\end{itemize}


\end{document}

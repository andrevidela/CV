%%%%%%%%%%%%%%%%%%%%%%%%%%%%%%%%%%%%%%%%%
% Plasmati Graduate CV
% LaTeX Template
% Version 1.0 (24/3/13)
%
% This template has been downloaded from:
% http://www.LaTeXTemplates.com
%
% Original author:
% Alessandro Plasmati (alessandro.plasmati@gmail.com)
%
% License:
% CC BY-NC-SA 3.0 ( )
%
% Important note:
% This template needs to be compiled with XeLaTeX.
% The main document font is called Fontin and can be downloaded for free
% from here: http://www.exljbris.com/fontin.html
%
%%%%%%%%%%%%%%%%%%%%%%%%%%%%%%%%%%%%%%%%%

%----------------------------------------------------------------------------------------
%	PACKAGES AND OTHER DOCUMENT CONFIGURATIONS
%----------------------------------------------------------------------------------------

\documentclass[a4paper,10pt]{article} % Default font size and paper size

\usepackage{fontspec} % For loading fonts
\defaultfontfeatures{Mapping=tex-text}

\usepackage{xunicode,xltxtra,url,parskip} % Formatting packages

\usepackage[usenames,dvipsnames]{xcolor} % Required for specifying custom colors

\usepackage[big]{layaureo} % Margin formatting of the A4 page, an alternative to layaureo can be \usepackage{fullpage}
% To reduce the height of the top margin uncomment: \addtolength{\voffset}{-1.3cm}

\usepackage{hyperref} % Required for adding links	and customizing them
\definecolor{linkcolour}{rgb}{0,0.2,0.6} % Link color
\hypersetup{colorlinks,breaklinks,urlcolor=linkcolour,linkcolor=linkcolour} % Set link colors throughout the document

\usepackage{titlesec} % Used to customize the \section command
\titleformat{\section}{\Large\scshape\raggedright}{}{0em}{}[\titlerule] % Text formatting of sections
\titlespacing{\section}{0pt}{3pt}{3pt} % Spacing around sections

\titleformat{\subsection}{\scshape\raggedright\large}{}{1em} {} []



\begin{document}

\pagestyle{empty} % Removes page numbering

%----------------------------------------------------------------------------------------
%	NAME AND CONTACT INFORMATION
%----------------------------------------------------------------------------------------

\par{\centering{\Huge André \textsc{Videla}}\bigskip\par} % Your name
\par{\centering{\textsc{iOS developper / Functional Programmer}}\bigskip\par} % Your name
\section{Personal Data}

\begin{tabular}{rl}
\textsc{Place and Date of Birth:} & Switzerland  | 20 August 1991 \\
\textsc{Address:} & Rue pré-du-marché 9B, 1004 Lausanne, Switzerland \\
\textsc{Phone:} & +41 76 822 0541\\
\textsc{email:} & \href{mailto:andre.videla@gmail.com}{andre.videla@gmail.com}
\end{tabular}

%----------------------------------------------------------------------------------------
%	Bio
%----------------------------------------------------------------------------------------

\section{Bio}
Fascinated by programming languages, I seek excellence, correctness and beauty in code. Lots of experiences in game development taught me to mix pragmatism and elegance. My deepest interests lie in type systems and programming language theory and design.

%----------------------------------------------------------------------------------------
%	WORK EXPERIENCE 
%----------------------------------------------------------------------------------------

\section{Professional Experience}

\begin{tabular}{r|p{9.3cm}|l}
\textsc{June 2017} & Software Engineer at Sicpa, Lausanne\\
\textsc{Oct. 2016} & Mobile engineer for iOS (CDD)\\ 
& \footnotesize{Designed, developed and maintained internal and public APIs and libraries for iOS. Developed and maintained small and medium scale iOS apps using agile Scrum.} & Swift, Obj-C\\
\\
\textsc{Mar. 2016}  & Release of \href{https://arcaea.lowiro.com}{Arcaea} for Android and iOS by lowiro (Remote)& \\
& \footnotesize{Designed and implemented a domain specific language to represent notes of a song in Arcea. The implementation has been done in C++ (Cocos2DX) for the game and Javascript for the web editor} & C++, ES6\\
\\
%------------------------------------------------
\textsc{Sept 2015} & Release of \href{https://itunes.apple.com/us/app/hackerspaces/id1035583993?ls=1&mt=8}{Hackerspaces} for iOS &\\
& \footnotesize{I developed all aspects of the application: prototyping, UI, UX, unit/UI testing, release, continuous integration and I kept supporting the app across 3 versions of swift.} & Swift\\
\\
\textsc{May 2015} & T-Aiko: A rhythm game on android and iOS (Remote) & \\
\textsc{May 2014} & \footnotesize{Collaborated with T-Aiko's developer to implement song parsing tools, real time event handling and UI} & Java\\
\\
\textsc{Feb. 2014} & Volunteer developer at Dischan (Remote)\\
\textsc{Sept 2012} & \footnotesize{Worked on Android port of the game engine including writing a parser for a DSL, asset management, audio management, 2D scene composition and rendering, UI management. Additionally I was part of the QA team for other projects.} & Java
\end{tabular}

%----------------------------------------------------------------------------------------
%	EDUCATION
%----------------------------------------------------------------------------------------

\section{Education}

\begin{tabular}{rl}	

July 2017 & EUTypes summer school \\&\footnotesize{Coq, Agda, HoTT, $\lambda$-calculus} \\
\multicolumn{2}{c}{}\\
May 2017 & Introduction to Logic by Stanford University on Coursera. \footnotesize{\href{https://www.coursera.org/account/accomplishments/certificate/RPGEPLA94HFF}{statement of accomplishment}}\\
\multicolumn{2}{c}{}\\
2012 - 2016 & Attended Bachelor in Computer Science at EPFL, Lausanne\\
& \footnotesize{Collected 131 credits}\\
\multicolumn{2}{c}{}\\

%------------------------------------------------

2007 - 2012 & Maturité Gymnasiale\\ & option principale Physique et Application des Math\\ & option secondaire Introduction à la programmation\\
\end{tabular}

\renewcommand{\arraystretch}{1.2}

%----------------------------------------------------------------------------------------
%	LANGUAGES
%----------------------------------------------------------------------------------------

\section{Languages}

\begin{tabular}{rl}
\textsc{French:} & Mothertongue\\

\textsc{English:} & Fluent\\

\textsc{Spanish:} & Good speaker\\

\textsc{Japanese:} & Basic knowledge\\

\end{tabular}

%----------------------------------------------------------------------------------------
%	Programming Languages
%----------------------------------------------------------------------------------------

\section{Programming Languages}

\begin{tabular}{r|l}
Mastery & Swift, Java\\
Excellent & Scala, JS, Obj-C\\
Intermediate & Idris, Haskell, Clojure, C++, C, C\#, \LaTeX\\
Beginner & Coq, Rust, Agda\\
\end{tabular}
%----------------------------------------------------------------------------------------
%	INTERESTS AND ACTIVITIES
%----------------------------------------------------------------------------------------

\section{Interests and Activities}
\begin{itemize}
	\item \textsc{Video games} My experience with competitive video games taught me new ways of interacting within a team, handling hard situations and keeping the right mindset in highly competitive settings. Nowadays, I keep a monthly "video game playing club" with a small group where we discuss the narrative value of interactive experiences.
	\item \textsc{Math} keeps me engaged in the latest developments in programming language research and theory. It allows me to find elegant and correct solutions to hard problems daily.
	\item \textsc{Sports} keep me healthy in many ways. I practice all sorts of individual trainings like running, fitness and swimming. I also played badminton with the team at my last company.
\end{itemize}


\end{document}

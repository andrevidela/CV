%%%%%%%%%%%%%%%%%%%%%%%%%%%%%%%%%%%%%%%%%
% Plasmati Graduate CV
% LaTeX Template
% Version 1.0 (24/3/13)
%
% This template has been downloaded from:
% http://www.LaTeXTemplates.com
%
% Original author:
% Alessandro Plasmati (alessandro.plasmati@gmail.com)
%
% License:
% CC BY-NC-SA 3.0 ( )
%
% Important note:
% This template needs to be compiled with XeLaTeX.
% The main document font is called Fontin and can be downloaded for free
% from here: http://www.exljbris.com/fontin.html
%
%%%%%%%%%%%%%%%%%%%%%%%%%%%%%%%%%%%%%%%%%

%----------------------------------------------------------------------------------------
%	PACKAGES AND OTHER DOCUMENT CONFIGURATIONS
%----------------------------------------------------------------------------------------

\documentclass[a4paper,10pt]{article} % Default font size and paper size

\usepackage{fontspec} % For loading fonts
\defaultfontfeatures{Mapping=tex-text}

\usepackage{xunicode,xltxtra,url,parskip} % Formatting packages

\usepackage[usenames,dvipsnames]{xcolor} % Required for specifying custom colors

\usepackage[big]{layaureo} % Margin formatting of the A4 page, an alternative to layaureo can be \usepackage{fullpage}
% To reduce the height of the top margin uncomment: \addtolength{\voffset}{-1.3cm}

\usepackage{hyperref} % Required for adding links	and customizing them
\definecolor{linkcolour}{rgb}{0,0.2,0.6} % Link color
\hypersetup{colorlinks,breaklinks,urlcolor=linkcolour,linkcolor=linkcolour} % Set link colors throughout the document

\usepackage{titlesec} % Used to customize the \section command
\titleformat{\section}{\Large\scshape\raggedright}{}{0em}{}[\titlerule] % Text formatting of sections
\titlespacing{\section}{0pt}{3pt}{3pt} % Spacing around sections

\titleformat{\subsection}{\scshape\raggedright\large}{}{1em} {} []

\usepackage{makecell} % Used to format cells in tables

\begin{document}

\pagestyle{empty} % Removes page numbering

%----------------------------------------------------------------------------------------
%	NAME AND CONTACT INFORMATION
%----------------------------------------------------------------------------------------

\par{\centering{\Huge André \textsc{Videla}}\bigskip\par} % Your name
\par{\centering{\textsc{Functional Programmer / Aspiring Type Theorist}}\bigskip\par} % Your name
\section{Personal Data}

\begin{tabular}{rl}
\textsc{Place and Date of Birth:} & Switzerland  | 20 August 1991 \\
\textsc{Address:} & Rue Gibraltar 12, 2000 Neuchatel, Switzerland \\
\textsc{Phone:} & +41 76 822 0541\\
\textsc{email:} & \href{mailto:andre.videla@gmail.com}{andre.videla@gmail.com}\\
\textsc{github:} & \href{https://github.com/andrevidela}{https://github.com/andrevidela}
\end{tabular}

%----------------------------------------------------------------------------------------
%	Bio
%----------------------------------------------------------------------------------------

\section{Bio}
Fascinated by programming languages, I seek excellence, correctness and beauty in code. I work a lot in Idris and Haskell trying to build real-world software out of provable programs. My deepest interests lie in type systems and programming language theory and design. \\In my free time I organise the \href{https://www.meetup.com/Formal-Methods-and-Verified-software-meetup}{\emph{Software verification meetup}} in Lausanne and I play video games.

%----------------------------------------------------------------------------------------
%	WORK EXPERIENCE 
%----------------------------------------------------------------------------------------

\section{Work Experience}
\renewcommand{\arraystretch}{1.3}
\begin{tabular}{r|p{9.3cm}|l}
\textsc{Apr. 2018} & Backend Developper at Bity SA (Neuchatel) & \\
\textsc{Dec. 2018} & \footnotesize{Lead a new project from scratch to production which aimed to provide a common API for interacting with cryptocurrency nodes. The goal was to enable easy management of additional cryptocurrencies and provide strong guarantees about the business logic by leveraging the Haskell type system and using thorough testing practices.} & Haskell\\

\makecell[cr]{
	\textsc{Oct. 2017 -} \\
	\textsc{Jan. 2018}
	} & \makecell[cl]{Mobile Developper at Krown SA (Geneva)\\
	    \footnotesize{Developed and maintained iOS app for business customers}}& Swift, Rx\\

\textsc{Oct 2016 -} & Software Engineer at Sicpa, Lausanne\\

\textsc{June. 2017}  
& \footnotesize{Designed, developed and maintained internal and public APIs and libraries for iOS. Developed and maintained small and medium scale iOS apps using agile Scrum.} 
& Swift, Obj-C\\

\textsc{Mar. 2016}  & Release of \href{https://arcaea.lowiro.com}{Arcaea} for Android and iOS by lowiro (Remote)& \\
&  \footnotesize{Designed and implemented a DSL to describe music notes in 3D and developed parsing and checking tools}& C++, Javascript\\

%------------------------------------------------

\textsc{Sept 2015 -}& Internship at \textsc{Kabotip}, Tokyo (Remote) \emph{}\\
\textsc{July 2015} & \footnotesize{Developed micro-services in Clojure for the Kabotip platform. Namely the mobile notification service and the automatic moderation tool for user comments.} & Clojure, Redis\\
\end{tabular}

%----------------------------------------------------------------------------------------
%	EDUCATION
%----------------------------------------------------------------------------------------

\section{Education}
\renewcommand{\arraystretch}{1.3}
\begin{tabular}{rp{11cm}}	
2017 - 2019 & Licence in Computer Science at Unidistance \footnotesize{(remote)}  \\ 

July 2017 & EUTypes summer school \footnotesize{(Coq, Agda, HoTT, $\lambda$-calculus)} \\
%\multicolumn{2}{c}{}\\
May 2017 & Introduction to Logic by Stanford University on Coursera. \footnotesize{\href{https://www.coursera.org/account/accomplishments/certificate/RPGEPLA94HFF}{statement of accomplishment}}\\
%\multicolumn{2}{c}{}\\
2012 - 2016 & Attended Bachelor in Computer Science at EPFL, Lausanne\\
& \footnotesize{Collected 131 credits}\\
%\multicolumn{2}{c}{}\\

%------------------------------------------------

2007 - 2012 & Maturité Gymnasiale \footnotesize{(physique et application des math)}
\end{tabular}

%----------------------------------------------------------------------------------------
%	SCHOLARSHIPS AND ADDITIONAL INFO
%----------------------------------------------------------------------------------------

\section{Lectures And Projects}
\renewcommand{\arraystretch}{1.5}
\begin{tabular}{rp{10cm}|l}
\textsc{Ongoing} & Reading \href{https://bentnib.org/quantitative-type-theory.html}{\emph{Quantitative Type Theory}} by Robert Atkey & \makecell[cl]{Type Theory, \\Idris 2}\\
\textsc{Feb.} 2018 & Interviewed for the Corecursive podcast: \href{https://corecursive.com/007-total-programming-using-swift-with-andre-videla}{007 - Total Programming Using Swift with Andre Videla} & Swift, Idris\\
\textsc{Dec.} 2017 & Gave a talk about Idris and dependent types at the Software verification meetup. & Idris\\
\textsc{Oct.} 2017 & Read \emph{State Machines All The Way Down} by Edwin Brady & Idris\\
\textsc{Sept.} 2017 & Read \emph{Kleisli arrows of outrageous fortune} by Conor McBride & Haskell (SHE)\\
\textsc{Aug.} 2017 & Presented work on \href{http://www.cse.chalmers.se/~nad/publications/danielsson-parser-combinators.html}{\emph{Total Parser Combinators}} by Nils Anders Danielsson for a small local conference (\href{https://www.youtube.com/watch?v=DzKVm6ApKFI}{VOD on youtube})& Idris, Agda\\

\end{tabular}
\renewcommand{\arraystretch}{1.2}

%----------------------------------------------------------------------------------------
%	LANGUAGES
%----------------------------------------------------------------------------------------

%----------------------------------------------------------------------------------------
%	COMPUTER SKILLS 
%----------------------------------------------------------------------------------------

\section{(Programming) Languages}

\begin{tabular}{lr}
		
	\begin{tabular}{r|l}
	Mastery & Swift, Java\\
	Excellent & Scala, JS, Obj-C, Haskell\\
	Intermediate & Idris, Clojure, C++, C, C\#\\
	Beginner & Coq, Rust, Agda\\
	\end{tabular}
&
	\begin{tabular}{rl}
	\textsc{French:} & Mothertongue\\
	
	\textsc{English:} & Fluent\\
	
	\textsc{Spanish:} & Good speaker\\
	
	\textsc{Japanese:} & Basic knowledge\\
	
	\end{tabular}
\end{tabular}


\end{document}

%%%%%%%%%%%%%%%%%%%%%%%%%%%%%%%%%%%%%%%%%
% Plasmati Graduate CV
% LaTeX Template
% Version 1.0 (24/3/13)
%
% This template has been downloaded from:
% http://www.LaTeXTemplates.com
%
% Original author:
% Alessandro Plasmati (alessandro.plasmati@gmail.com)
%
% License:
% CC BY-NC-SA 3.0 ( )
%
% Important note:
% This template needs to be compiled with XeLaTeX.
% The main document font is called Fontin and can be downloaded for free
% from here: http://www.exljbris.com/fontin.html
%
%%%%%%%%%%%%%%%%%%%%%%%%%%%%%%%%%%%%%%%%%

%----------------------------------------------------------------------------------------
%	PACKAGES AND OTHER DOCUMENT CONFIGURATIONS
%----------------------------------------------------------------------------------------

\documentclass[a4paper,10pt]{article} % Default font size and paper size

\usepackage{fontspec} % For loading fonts
\defaultfontfeatures{Mapping=tex-text}

\usepackage{xunicode,xltxtra,url,parskip} % Formatting packages

\usepackage[usenames,dvipsnames]{xcolor} % Required for specifying custom colors

\usepackage[big]{layaureo} % Margin formatting of the A4 page, an alternative to layaureo can be \usepackage{fullpage}
% To reduce the height of the top margin uncomment: \addtolength{\voffset}{-1.3cm}

\usepackage{hyperref} % Required for adding links	and customizing them
\definecolor{linkcolour}{rgb}{0,0.2,0.6} % Link color
\hypersetup{colorlinks,breaklinks,urlcolor=linkcolour,linkcolor=linkcolour} % Set link colors throughout the document

\usepackage{titlesec} % Used to customize the \section command
\titleformat{\section}{\Large\scshape\raggedright}{}{0em}{}[\titlerule] % Text formatting of sections
\titlespacing{\section}{0pt}{3pt}{3pt} % Spacing around sections

\titleformat{\subsection}{\scshape\raggedright\large}{}{1em} {} []

\usepackage{makecell} % Used to format cells in tables

\begin{document}

\pagestyle{empty} % Removes page numbering

%----------------------------------------------------------------------------------------
%	NAME AND CONTACT INFORMATION
%----------------------------------------------------------------------------------------

\par{\centering{\Huge André \textsc{Videla}}\bigskip\par} % Your name
\par{\centering{\textsc{Functional Programmer / Aspiring Type Theorist}}\bigskip\par} % Your name
\section{Personal Data}

\begin{tabular}{rl}
\textsc{Place and Date of Birth:} & Switzerland  | 20 August 1991 \\
\textsc{Address:} & Rue Gibraltar 12, 2000 Neuchatel, Switzerland \\
\textsc{Phone:} & +41 76 822 0541\\
\textsc{email:} & \href{mailto:andre.videla@gmail.com}{andre.videla@gmail.com}
\end{tabular}

%----------------------------------------------------------------------------------------
%	Bio
%----------------------------------------------------------------------------------------

\section{Bio}
Fascinated by programming languages, I seek excellence, correctness and beauty in code. Currently I work a lot in Idris trying to build real-world software out of provable programs. My deepest interests lie in type systems and programming language theory and design.

%----------------------------------------------------------------------------------------
%	WORK EXPERIENCE 
%----------------------------------------------------------------------------------------

\section{Work Experience}
\renewcommand{\arraystretch}{1.3}
\begin{tabular}{r|p{9.3cm}|l}
\textsc{Apr. 2018} & Backend Developper at Bity SA (Neuchatel) & \\
\textsc{Dec. 2018} & \footnotesize{Developed services to provide a common API to interact with cryptocurrency nodes. Lead the architectural effort to enable easy extension using the Haskell type system and provided strong guarantees about the business logic through thorough testing and type information.} & Haskell\\

\makecell[cr]{
	\textsc{Oct. 2017 -} \\
	\textsc{Jan. 2018}
	} & \makecell[cl]{Mobile Developper at Krown SA (Geneva)\\
	    \footnotesize{Developed and maintained iOS app for business customers}}& Swift, Rx\\

\textsc{Oct 2016 -} & Software Engineer at Sicpa, Lausanne\\

\textsc{June. 2017}  
& \footnotesize{Designed, developed and maintained internal and public APIs and libraries for iOS. Developed and maintained small and medium scale iOS apps using agile Scrum.} 
& Swift, Obj-C\\

\textsc{Mar. 2016}  & Release of \href{https://arcaea.lowiro.com}{Arcaea} for Android and iOS by lowiro (Remote)& \\
&  \footnotesize{Designed and implemented a DSL to describe music notes in 3D, including parsing and checking tools}& C++, Javascript\\

%------------------------------------------------

\textsc{Sept 2015 -}& Internship at \textsc{Kabotip}, Tokyo (Remote) \emph{}\\
\textsc{July 2015} & \footnotesize{Developed micro-services in Clojure for the Kabotip platform. Namely the mobile notification service and the automatic moderation tool for user comments.} & Clojure, Redis\\
\textsc{May} 2015 & T-Aiko: A rhythm game on android and iOS (Remote) & Java\\
& \footnotesize{Retro engineer existing DSL and wrote parsing tools and interpreter. As well as other tasks such as UI and real-time event handling} & \\

\end{tabular}

%----------------------------------------------------------------------------------------
%	EDUCATION
%----------------------------------------------------------------------------------------

\section{Education}
\renewcommand{\arraystretch}{1.3}
\begin{tabular}{rl}	
Oct. 2017 - 2019 & Licence in Computer Science at Unidistance  \\ 

July 2017 & EUTypes summer school \\&\footnotesize{Coq, Agda, HoTT, $\lambda$-calculus} \\
%\multicolumn{2}{c}{}\\
May 2017 & Introduction to Logic by Stanford University on Coursera. \footnotesize{\href{https://www.coursera.org/account/accomplishments/certificate/RPGEPLA94HFF}{statement of accomplishment}}\\
%\multicolumn{2}{c}{}\\
2012 - 2016 & Attended Bachelor in Computer Science at EPFL, Lausanne\\
& \footnotesize{Collected 131 credits}\\
%\multicolumn{2}{c}{}\\

%------------------------------------------------

2007 - 2012 & Maturité Gymnasiale\\ & option principale Physique et Application des Math\\ & option secondaire Introduction à la programmation\\
\end{tabular}

%----------------------------------------------------------------------------------------
%	SCHOLARSHIPS AND ADDITIONAL INFO
%----------------------------------------------------------------------------------------

\section{Lectures And Projects}
\renewcommand{\arraystretch}{1.5}
\begin{tabular}{rp{10cm}|l}
\textsc{Ongoing} & Organiser of the \href{https://www.meetup.com/Formal-Methods-and-Verified-software-meetup}{\emph{Software verification meetup}} in Lausanne since 2017. & \\
\textsc{Ongoing} & Reading \href{https://bentnib.org/quantitative-type-theory.html}{\emph{Quantitative Type Theory}} by Robert Atkey & \makecell[cl]{Type Theory, \\Idris 2}\\
\textsc{Feb.} 2018 & Interviewed for the Corecursive podcast: \href{https://corecursive.com/007-total-programming-using-swift-with-andre-videla}{007 - Total Programming Using Swift with Andre Videla} & Swift, Idris\\
\textsc{Dec.} 2017 & Gave a talk about Idris and dependent types at the Software verification meetup. & Idris\\
\textsc{Oct.} 2017 & Read \emph{State Machines All The Way Down} by Edwin Brady & Idris\\
\textsc{Sept.} 2017 & Read \emph{Kleisli arrows of outrageous fortune} by Conor McBride & Haskell (SHE)\\
\textsc{Aug.} 2017 & Presented work on \href{http://www.cse.chalmers.se/~nad/publications/danielsson-parser-combinators.html}{\emph{Total Parser Combinators}} by Nils Anders Danielsson for a small local conference (\href{https://www.youtube.com/watch?v=DzKVm6ApKFI}{VOD on youtube})& Idris, Agda\\
\textsc{Sept 2015} & Release of \href{https://itunes.apple.com/us/app/hackerspaces/id1035583993?ls=1&mt=8}{Hackerspaces} for iOS & Swift\\

\end{tabular}
\renewcommand{\arraystretch}{1.2}

%----------------------------------------------------------------------------------------
%	LANGUAGES
%----------------------------------------------------------------------------------------

\section{Languages}

\begin{tabular}{rl}
\textsc{French:} & Mothertongue\\

\textsc{English:} & Fluent\\

\textsc{Spanish:} & Good speaker\\

\textsc{Japanese:} & Basic knowledge\\

\end{tabular}

%----------------------------------------------------------------------------------------
%	COMPUTER SKILLS 
%----------------------------------------------------------------------------------------

\section{Programming Languages}

\begin{tabular}{r|l}
Mastery & Swift, Java\\
Excellent & Scala, JS, Obj-C, Haskell\\
Intermediate & Idris, Clojure, C++, C, C\#\\
Beginner & Coq, Rust, Agda\\
\end{tabular}
%----------------------------------------------------------------------------------------
%	INTERESTS AND ACTIVITIES
%----------------------------------------------------------------------------------------

\section{Interests and Activities}
\begin{itemize}
	\item \textsc{Video games} continuously teach me new ways of interacting within a team, handling hard situations and keeping the right mindset in highly competitive settings. I also keep a monthly "video game playing club" with a small group where we discuss the narrative value of interactive experiences.
	\item \textsc{Math} keeps me engaged in the latest developments in programming language research and theory. It allows me to find elegant and correct solutions to hard problems daily.
	\item \textsc{Sports} allow me to stay healthy many ways. I practice all sorts of individual trainings like running, fitness exercises and swimming. I also played badminton with the team at Sicpa SA.
\end{itemize}


\end{document}

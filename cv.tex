%%%%%%%%%%%%%%%%%%%%%%%%%%%%%%%%%%%%%%%%%
% Plasmati Graduate CV
% LaTeX Template
% Version 1.0 (24/3/13)
%
% This template has been downloaded from:
% http://www.LaTeXTemplates.com
%
% Original author:
% Alessandro Plasmati (alessandro.plasmati@gmail.com)
%
% License:
% CC BY-NC-SA 3.0 ( )
%
% Important note:
% This template needs to be compiled with XeLaTeX.
% The main document font is called Fontin and can be downloaded for free
% from here: http://www.exljbris.com/fontin.html
%
%%%%%%%%%%%%%%%%%%%%%%%%%%%%%%%%%%%%%%%%%

%----------------------------------------------------------------------------------------
%	PACKAGES AND OTHER DOCUMENT CONFIGURATIONS
%----------------------------------------------------------------------------------------

\documentclass[a4paper,10pt]{article} % Default font size and paper size

\usepackage{fontspec} % For loading fonts
\defaultfontfeatures{Mapping=tex-text}

\usepackage{xunicode,xltxtra,url,parskip} % Formatting packages

\usepackage[usenames,dvipsnames]{xcolor} % Required for specifying custom colors

\usepackage[big]{layaureo} % Margin formatting of the A4 page, an alternative to layaureo can be \usepackage{fullpage}
% To reduce the height of the top margin uncomment: \addtolength{\voffset}{-1.3cm}

\usepackage{hyperref} % Required for adding links	and customizing them
\definecolor{linkcolour}{rgb}{0,0.2,0.6} % Link color
\hypersetup{colorlinks,breaklinks,urlcolor=linkcolour,linkcolor=linkcolour} % Set link colors throughout the document

\usepackage{titlesec} % Used to customize the \section command
\titleformat{\section}{\Large\scshape\raggedright}{}{0em}{}[\titlerule] % Text formatting of sections
\titlespacing{\section}{0pt}{3pt}{3pt} % Spacing around sections

\titleformat{\subsection}{\scshape\raggedright\large}{}{1em} {} []



\begin{document}

\pagestyle{empty} % Removes page numbering

%----------------------------------------------------------------------------------------
%	NAME AND CONTACT INFORMATION
%----------------------------------------------------------------------------------------

\par{\centering{\Huge André \textsc{Videla}}\bigskip\par} % Your name
\par{\centering{\textsc{Functional Programmer / Aspiring Type Theorist}}\bigskip\par} % Your name
\section{Personal Data}

\begin{tabular}{rl}
\textsc{Place and Date of Birth:} & Switzerland  | 20 August 1991 \\
\textsc{Address:} & Rue pré-du-marché 9B, 1004 Lausanne, Switzerland \\
\textsc{Phone:} & +41 79 822 0541\\
\textsc{email:} & \href{mailto:andre.videla@gmail.com}{andre.videla@gmail.com}
\end{tabular}

%----------------------------------------------------------------------------------------
%	Bio
%----------------------------------------------------------------------------------------

\section{Bio}
Fascinated by programming languages, I seek excellence, correctness and beauty in code. Lots of experiences in game development taught me to mix pragmatism and elegance. My deepest interests lie in type systems and programming language theory and design.

%----------------------------------------------------------------------------------------
%	WORK EXPERIENCE 
%----------------------------------------------------------------------------------------

\section{Competencies}

\subsection{Functional Programming}

\begin{tabular}{r|p{9.3cm}|l}

\textsc{Aug.} 2017 & Programming with dependent types: Completed TTD in Idris by Edwin Brady with exercises available on \href{https://github.com/andrevidela/Type_Driven_Dev-Idris}{github} & Idris\\\\

\textsc{Mar.} 2016 & Designed and implemented a music description language for \href{https://arcaea.lowiro.com}{Arcaea} (Remote) & C++, Javascript\\\\

%------------------------------------------------

\textsc{Sept 2015} & Internship at \textsc{Kabotip}, Tokyo (Remote) \emph{}\\
\textsc{July 2015} & \footnotesize{Developed micro-services in Clojure for the Kabotip platform. Namely the mobile notification service and the automatic moderation tool for user comments.} & Clojure, Redis\\\\

\textsc{June 2015} & JS-Go: A pure, stateless game of Go in Scala & Scala, Scala-JS\\

\end{tabular}

\subsection{Mobile Development}

\begin{tabular}{r|p{9.3cm}|l}
\textsc{June 2017} & Software Engineer at Sicpa, Lausanne\\
\textsc{Oct. 2016} & Mobile engineer for iOS\\ 
& \footnotesize{Designed, developed and maintained internal and public APIs and libraries for iOS. Developed and maintained small and medium scale mobile applications.} & Swift, Obj-C\\
\\
%------------------------------------------------
\textsc{Sept} 2015 & \href{https://itunes.apple.com/us/app/hackerspaces/id1035583993?ls=1&mt=8}{Hackerspaces} for iOS: an App that allows you to track hackerspace openings across the world & Swift\\\\

\textsc{May} 2015 & T-Aiko: A rhythm game on android and iOS (Remote) & \\
& \footnotesize{Implement parsing tools, real time event handling and UI} & Java\\\\

\textsc{2013-2014} & Volunteer developer at Dischan, (Remote)\\
& \footnotesize{Worked on Android ports and helped with QA} & Java
\end{tabular}

%----------------------------------------------------------------------------------------
%	EDUCATION
%----------------------------------------------------------------------------------------

\section{Education}

\begin{tabular}{rl}	

July 2017 & EUTypes summer school \\&\footnotesize{Coq, Agda, HoTT, $\lambda$-calculus} \\
\multicolumn{2}{c}{}\\
May 2017 & Introduction to Logic by Stanford University on Coursera. \footnotesize{\href{https://www.coursera.org/account/accomplishments/certificate/RPGEPLA94HFF}{statement of accomplishment}}\\
\multicolumn{2}{c}{}\\
2012 - 2016 & Attended Bachelor in Computer Science at EPFL, Lausanne\\
& \footnotesize{Collected 131 credits}\\
\multicolumn{2}{c}{}\\

%------------------------------------------------

2007 - 2012 & Maturité Gymnasiale\\ & option principale Physique et Application des Math\\ & option secondaire Introduction à la programmation\\
\end{tabular}

%----------------------------------------------------------------------------------------
%	SCHOLARSHIPS AND ADDITIONAL INFO
%----------------------------------------------------------------------------------------

\section{Miscellaneous Projects}
\renewcommand{\arraystretch}{1.5}
\begin{tabular}{rp{10cm}|l}
\textsc{Aug.} 2017 & Presented work on \href{http://www.cse.chalmers.se/~nad/publications/danielsson-parser-combinators.html}{\emph{Total Parser Combinators}} by Nils Anders Danielsson for a small conference & Swift\\
\textsc{July} 2016 & \href{http://epicgamejam.com/game/delta-zone}{Delta-zone}, a color-based puzzle game made for the Epic Game Jam & Unity and C\#\\
\textsc{Apr.} 2015 & Facebook Hackaton @EPFL. A project to take care of a connected plant. & Swift\\

\end{tabular}
\renewcommand{\arraystretch}{1.2}

%----------------------------------------------------------------------------------------
%	LANGUAGES
%----------------------------------------------------------------------------------------

\section{Languages}

\begin{tabular}{rl}
\textsc{French:} & Mothertongue\\

\textsc{English:} & Fluent\\

\textsc{Spanish:} & Good speaker\\

\textsc{Japanese:} & Basic knowledge\\

\end{tabular}

%----------------------------------------------------------------------------------------
%	COMPUTER SKILLS 
%----------------------------------------------------------------------------------------

\section{Programming Languages}

\begin{tabular}{r|l}
Mastery & Swift, Java\\
Excellent & Scala, JS, Obj-C\\
Intermediate & Idris, Haskell, Clojure, C++, C, C\#\\
Beginner & Coq, Rust, Agda\\
\end{tabular}
%----------------------------------------------------------------------------------------
%	INTERESTS AND ACTIVITIES
%----------------------------------------------------------------------------------------

\section{Interests and Activities}
\begin{itemize}
	\item \textsc{Video games} continuously teach me new ways of interacting within a team, handling hard situations and keeping the right mindset in highly competitive settings. I also keep a monthly "video game playing club" with a small group where we discuss the narrative value of interactive experiences.
	\item \textsc{Math} keeps me engaged in the latest developments in programming language research and theory. It allows me to find elegant and correct solutions to hard problems daily.
	\item \textsc{Sports} keep my healthy in many ways. I practice all sorts of individual trainings like running, fitness and swimming. I also played badminton with the team at my last company.
\end{itemize}


\end{document}
